\vspace*{15em}

%\begin{center}
\chapter*{Agradecimientos}

Aún tengo muy presentes los inicios de esta tesis doctoral. Allá por el 2018, recién comenzada mi etapa profesional en Telefónica, me encontraba con el deseo de llevar a cabo una tesis doctoral en Inteligencia Artificial con la clara motivación de fondo de que pudiese contribuir con ello al bien común y ayudase a evitar los futuros distópicos tan comunes en muchas obras de ciencia ficción. Esto me sirvió para descubrir el ámbito de la Inteligencia Artificial Explicable y ver que era en esa dirección en la que quería profundizar y buscar maneras de poder contribuir yo de manera activa a ese campo del conocimiento. 

Ahora bien, en torno a esta cuestión tenía más preguntas que respuestas. ¿Cómo realizar un doctorado?, ¿podría vincular esto a problemas de negocio reales de la empresa?... Es aquí donde mi jefe por aquel entonces, Pedro Antonio Alonso Baigorri, me recomendó hablar con Richard Benjamins para poder llevar a cabo la tesis doctoral... y ahí comenzó todo. Richard se ofreció a dirigir mi tesis desde Telefónica, y me presentó a Óscar Corcho para que dirigiese mi tesis desde la Universidad Politécnica de Madrid.

Tengo que agradecer mucho a mis dos tutores, Richard y Óscar, por todo el trabajo, tiempo, y dedicación que han puesto en mi tesis doctoral, y lo mucho que me han ayudado durante este proceso. En concreto, me siento agradecido a Richard por ser para mi un referente en cómo combinar el mundo académico con el mundo empresarial, y ver cómo poder llevar a cabo investigación aplicada que tenga un impacto tangible en la industria. A su vez, quiero darle las gracias porque me ha ayudado a conocer a grandes investigadores del sector (Francisco Herrera, Javier del Ser o Natalia Díaz-Rodríguez) de los que también he aprendido mucho. También quiero expresar mi profundo agradecimiento a Óscar por lo mucho que me ha ayudado durante la tesis, no sólo en términos relacionados con la investigación, sino también en el ámbito más humano. Durante todas las dificultades que han ido surgiendo, él siempre ha sido un gran apoyo, y constantemente me ha ayudado a ver como encontrar soluciones para resolver esos problemas. Sin él, sin duda, esta tesis no hubiese sido posible. 

Quiero agradecer también a todos los compañeros de Telefónica con los que trabajé durante esta tesis doctoral. En particular, a mi anterior jefe, Pedro, porque gracias a él comencé esta aventura, y por darme la oportunidad de vincular la investigación a los productos con los que trabajábamos día a día. También agradezco a mis compañeros de entonces, Federico Pérez Rosado, Raquel Crespo Crisenti, Daniel García Fernández y Álvaro Sánchez Pérez, por ser para mi un referente profesional, y por todas las conversaciones que hemos tenido, muchas de las cuales han ayudado también con aspectos relacionados con esta tesis.

Junto a esto, agradezco mucho a mi familia por todo el apoyo que me ha dado siempre, y en particular durante estos años de trabajo de la tesis doctoral. Agradezco a mi madre Lola y a mi padre Juan por confiar siempre en mi, por animarme a realizar la tesis, y por todo lo que he aprendido de ellos a lo largo de mi vida. Agradezco muy especialmente a mi esposa, Débora, por haber sido el gran soporte durante este tiempo, por haber estado conmigo día a día durante todo el camino de la tesis doctoral, y por ver siempre en mi lo mejor pase lo que pase. También agradezco el apoyo de mi hermana Victoria, y del resto de mi familia, en concreto de mis suegros Antonio e Inmaculada.

% Jesus
Finalmente, quiero dar las gracias a quien ha sido y es el mayor apoyo en todo lo que hago, a Jesús, quien dio una orientación definitiva a mi vida y de quien han venido todas las bendiciones que he recibido. Doy también las gracias a la Virgen María, a quien he podido acudir siempre en busca de apoyo y consuelo. \\

\textit{Ad maiorem Dei gloriam}

%\end{center}

\phantomsection
\addcontentsline{toc}{chapter}{Agradecimientos}

\newpage